%%%%%%%%%%%%%%%%%%%%%%%%%%%%%%%%%%%%%%%%%%%%%%%%%%%%%%%%%
%%%  PyObject
%%%%%%%%%%%%%%%%%%%%%%%%%%%%%%%%%%%%%%%%%%%%%%%%%%%%%%%%%
\chapter{Python 中的对象}



\section{Python 中对象的概念}
Python 中一切都是对象。在 Python 中,所有东西都是对象,整数,字符串,类这些都是对象。 在平时,我们常常会写类似下面这样子的一些代码。

\begin{lstlisting}[language=Python, numbers=left, numbersep=1em, numberstyle=\footnotesize , breaklines=true]
def Foo(object):
	pass
	
foo = Foo()
n = 10
\end{lstlisting}

其中 Foo 类,Foo 继承的 object 类,Foo 的实例 foo,代表整数 10 的 n,这些都是对象。在 Python 中,整数是一个对象,浮点数是一个对象,字符串是一个对象,列表、字典等等都是对象。int、str、dict 和用户自定义的类,可以表示类型,被称为类型对象。类型对象实例话之后的到对象,例如上面的 foo,被称为实例对象。

在开始介绍 Python 的内建对象之前,先介绍一下 PyObject。这个对象是 Python 中所有对象的基础。在 object.h 文件中介绍了 Python 对象的
设计的基本原则。

\begin{lstlisting}[language=C, numbers=left, numbersep=1em, numberstyle=\footnotesize , breaklines=true]
// Include/object.h
/*
Objects are structures allocated on the heap.  Special rules apply to
the use of objects to ensure they are properly garbage-collected.
Objects are never allocated statically or on the stack; they must be
accessed through special macros and functions only.  (Type objects are
exceptions to the first rule; the standard types are represented by
statically initialized type objects, although work on type/class unification
for Python 2.2 made it possible to have heap-allocated type objects too).

An object has a 'reference count' that is increased or decreased when a
pointer to the object is copied or deleted; when the reference count
reaches zero there are no references to the object left and it can be
removed from the heap.

An object has a 'type' that determines what it represents and what kind
of data it contains.  An object's type is fixed when it is created.
Types themselves are represented as objects; an object contains a
pointer to the corresponding type object.  The type itself has a type
pointer pointing to the object representing the type 'type', which
contains a pointer to itself!.

Objects do not float around in memory; once allocated an object keeps
the same size and address.  Objects that must hold variable-size data
can contain pointers to variable-size parts of the object.  Not all
objects of the same type have the same size; but the size cannot change
after allocation.  (These restrictions are made so a reference to an
object can be simply a pointer -- moving an object would require
updating all the pointers, and changing an object's size would require
moving it if there was another object right next to it.)

Objects are always accessed through pointers of the type 'PyObject *'.
The type 'PyObject' is a structure that only contains the reference count
and the type pointer.  The actual memory allocated for an object
contains other data that can only be accessed after casting the pointer
to a pointer to a longer structure type.  This longer type must start
with the reference count and type fields; the macro PyObject_HEAD should be
used for this (to accommodate for future changes).  The implementation
of a particular object type can cast the object pointer to the proper
type and back.

A standard interface exists for objects that contain an array of items
whose size is determined when the object is allocated.
*/

/* 翻译
对象是在堆上分配的结构。一些特殊规则被用在对象上以确保它们被正确地垃圾回收。
1. 对象从不被静态分配或在栈上分配; 它们只能通过特殊的宏和函数访问。(类型对象是
第一条规则的例外; 标准类型由静态初始化的类型对象表示,尽管 Python 2.2 中关于
类型/类统一的工作也使堆分配类型对象成为可能。)

2. 对象都有一个“引用计数”,当指向该对象的指针被复制或删除时,引用计数会增加或
减少;当引用计数达到0时,就没有对该对象的引用了,可以将其从堆中移除。

3. 对象都有一个“类型”,类型决定了对象代表什么以及对象包含什么类型的数据。
对象的类型在创建时是固定的。类型本身也被表示为对象; 对象包含指向相应类型对象的指针。
类型对象本身有一个类型指针,指向表示类型 'type' 的对象,该对象包含一个指向自身的指针!

4. 对象的地址不会发生变化; 对象一旦被分配,大小和内存地址都不会发生改变。保存可变大小
数据的对象会包含指针,指向代表对象可变大小的那部分。并非所有相同类型的对象都具有相
同的大小。这些限制使得只需要使用一个指针就可以引用一个对象。而如果要移动对象,则需
要更新所有指针。如果需要改变一个对象对象的大小则可能需要移动这个对象,因为在这个对
象被分配的内存地址的后方可能正好存在另一个对象。

5. 对象总是通过 'PyObject *' 类型的指针来访问。类型 'PyObject' 是一个只包含引用计数和
类型指针的结构体。只有在将指针转换为具体类型对象指针之后才能访问为对象分配的其他数据。
具体类型对象指针必须放在 包含 "引用计数" 字段和 "类型" 字段的 'PyObject_HEAD' 字段后面。
具体类型对象指针可以将对象指针强制转换为正确的类型并返回。

6. 对于包含项目数组的对象存在标准接口,这些项目的大小在分配对象时确定。
*/
\end{lstlisting}

\section{PyObject 和 PyVarObject 的实现}
在本节中,我们通过研究代码来看一下上一节中提到的 PyObject 到底是何方神圣。

\begin{lstlisting}[language=C, numbers=left, numbersep=1em, numberstyle=\footnotesize , breaklines=true]
// Include/object.h
// 只有编译的时候带了 with_trace_refs 参数才会打开这个开关,默认是关闭的
#ifdef Py_TRACE_REFS
/* Define pointers to support a doubly-linked list of all live heap objects. */
#define _PyObject_HEAD_EXTRA            \
    struct _object *_ob_next;           \
    struct _object *_ob_prev;

#define _PyObject_EXTRA_INIT 0, 0,

#else
#  define _PyObject_HEAD_EXTRA
#  define _PyObject_EXTRA_INIT
#endif

/* PyObject_HEAD defines the initial segment of every PyObject. */
#define PyObject_HEAD                   PyObject ob_base;

#define PyObject_HEAD_INIT(type)        \
    { _PyObject_EXTRA_INIT              \
    1, type },

#define PyVarObject_HEAD_INIT(type, size)       \
    { PyObject_HEAD_INIT(type) size },

/* PyObject_VAR_HEAD defines the initial segment of all variable-size
 * container objects.  These end with a declaration of an array with 1
 * element, but enough space is malloc'ed so that the array actually
 * has room for ob_size elements.  Note that ob_size is an element count,
 * not necessarily a byte count.
 */
#define PyObject_VAR_HEAD      PyVarObject ob_base;
#define Py_INVALID_SIZE (Py_ssize_t)-1

/* Nothing is actually declared to be a PyObject, but every pointer to
 * a Python object can be cast to a PyObject*.  This is inheritance built
 * by hand.  Similarly every pointer to a variable-size Python object can,
 * in addition, be cast to PyVarObject*.
 */
typedef struct _object {
    _PyObject_HEAD_EXTRA // 调试使用,在正常编译的时候,这一项为空
    Py_ssize_t ob_refcnt;  // 存储对象的引用计数
    PyTypeObject *ob_type; // 存储对象的实际类型
} PyObject;

typedef struct {
    PyObject ob_base;
    Py_ssize_t ob_size;  // 可变部分的元素数量
} PyVarObject;
\end{lstlisting}

上面的代码中,需要注意的还包括 PyObject\_HEAD 和 PyObject\_VAR\_HEAD 这两个宏定义。
其中 PyObject\_HEAD 表示 Python 中不可变大小对象的头部,使用 PyObject 实现, PyObject\_VAR\_HEAD 表示可变大小对象的头部,
使用 PyVarObject 实现。光说这个可能不是很理解,我们直接在 Python 源码中找两个例子来进行说明。

\begin{lstlisting}[language=C, numbers=left, numbersep=1em, numberstyle=\footnotesize , breaklines=true]
typedef struct {
    PyObject_HEAD
    double ob_fval;
} PyFloatObject;

typedef struct {
    PyObject_VAR_HEAD
    /* Vector of pointers to list elements.  list[0] is ob_item[0], etc. */
    PyObject **ob_item;
    Py_ssize_t allocated;
} PyListObject;
\end{lstlisting}

PyFloatObject 表示的是 Python 中的浮点数对象,其中使用了 PyObject\_HEAD,这个和我们的预期一致,浮点数就应该是一个内存大小
不可变的对象。PyListObject 表示 Python 中的列表对象,使用 PyObject\_VAR\_HEAD 来填充头部,这个和我们的直觉也保持一致,列表
对象中元素的个数是可变的,所以列表对象理应用一个可变大小的对象来实现。好了,搞清楚了 PyObject 和 PyVarObject 这两个东西之后,接下来就正式进入对 Python 对象的研究。

\section{类型对象}

前面说过,Python 中的所有对象都是由类实例化而来,这个类就表示对象的类型, 常用的 int, float, str 等等都是类型,而上面提到过,Python 中一切皆对象。那么 int、float、str 作为一个对象,他们的类型又是什么呢?我们可以通过 Python 的内建函数 type 来查看一个任意一个对象的类型。

\begin{lstlisting}[language=Python, numbers=left, numbersep=1em, numberstyle=\footnotesize , breaklines=true]
>>> type(int)
<class 'type'>
>>> type(float)
<class 'type'>
>>> type(str)
<class 'type'>
>>> type(dict)
<class 'type'>
\end{lstlisting}

可以看到,int,float,dict 等 Python 内建类型对象的类型都是 type。可以再看一下用户定义类的类型是什么。

\begin{lstlisting}[language=Python, numbers=left, numbersep=1em, numberstyle=\footnotesize , breaklines=true]
>>> class A(object):
...     pass
...
>>> type(A)
<class 'type'>
>>>
>>> class B(A):
...     pass
...
>>> type(B)
<class 'type'>
\end{lstlisting}

不管是我们定义的继承 object 的类 A 还是继承类 A 的类 B,它们的类型都是 type。type 类型被称为 Python 中的元类。
整个 Python 的对象体系都构建在 type 之上。type 作为一个类对象,它的类型也是 type,即 type 的类型为他自己。

\begin{lstlisting}[language=Python, numbers=left, numbersep=1em, numberstyle=\footnotesize , breaklines=true]
>>> type(type)
<class 'type'>
\end{lstlisting}


TODO:此处补一张类型之间关系的图片

PyTypeObject 就是 type 类型的实现。看一下它是怎么定义的。

\begin{lstlisting}[language=C, numbers=left, numbersep=1em, numberstyle=\footnotesize , breaklines=true]
typedef struct _typeobject {
    PyObject_VAR_HEAD
    // 定义了一个类型的名字
    const char *tp_name; /* For printing, in format "<module>.<name>" */
    Py_ssize_t tp_basicsize, tp_itemsize; /* For allocation */

    /* Methods to implement standard operations */

    destructor tp_dealloc;
    Py_ssize_t tp_vectorcall_offset;
    // 定义类型的存取函数
    getattrfunc tp_getattr;
    setattrfunc tp_setattr;
    PyAsyncMethods *tp_as_async; /* formerly known as tp_compare (Python 2)
                                    or tp_reserved (Python 3) */
    // 类型的 __repr__ 属性函数
    reprfunc tp_repr;

    /* Method suites for standard classes */
    // 标准的类型函数,一个 class 可以是 number、sequence 或者 mapping 类型
    PyNumberMethods *tp_as_number;
    PySequenceMethods *tp_as_sequence;
    PyMappingMethods *tp_as_mapping;

    /* More standard operations (here for binary compatibility) */

    hashfunc tp_hash;
    ternaryfunc tp_call;
    reprfunc tp_str;
    getattrofunc tp_getattro;
    setattrofunc tp_setattro;
	// ... ...
    /* Attribute descriptor and subclassing stuff */
    struct PyMethodDef *tp_methods;
    struct PyMemberDef *tp_members;
    struct PyGetSetDef *tp_getset;
    // Strong reference on a heap type, borrowed reference on a static type
    struct _typeobject *tp_base;
    PyObject *tp_dict;
    descrgetfunc tp_descr_get;
    descrsetfunc tp_descr_set;
    Py_ssize_t tp_dictoffset;
    initproc tp_init;
    allocfunc tp_alloc;
    newfunc tp_new;
    freefunc tp_free; /* Low-level free-memory routine */
    inquiry tp_is_gc; /* For PyObject_IS_GC */
    // 对象的基类列表
    PyObject *tp_bases;
    // 对象的 MRO 列表
    PyObject *tp_mro; /* method resolution order */
    PyObject *tp_cache;
    PyObject *tp_subclasses;
    PyObject *tp_weaklist;
    destructor tp_del;
    // ... ...
} PyTypeObject;
\end{lstlisting}

上面省略了暂时不需要研究的一些成员、剩下的成员都需要我们重点关注,这些成员对于我们理解 Python 的类型系统有很大的帮助。

\subsection{对象是如何继承一个}

\section{实例对象}

\section{Python 定义一个对象的时候是如何确定它的类型的}


























