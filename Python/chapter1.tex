\chapter{环境准备}

本书对 Python 3.10.4 版本的代码实现进行探究。使用的 Python 实现为 CPython。

\section{Python 编译}

为了了解 Python 的细节,探究和验证代码的实现是否符合自己的预期,我们需要亲自动手修改源码,添加测试代码。
为了完成这个目的,需要学会自己动手编译 Python 源码。所以在本节,将介绍如何编译 Python 代码。

\subsection{在 Linux/MacOS 平台上编译 Python}

编译步骤如下:

第一步是下载 Python 代码并切换到 Python 3.10.4 版本。

\begin{lstlisting}[language=bash, numbers=left, numbersep=1em, numberstyle=\footnotesize , breaklines=true]
# 安装开发包(非必需)
$ apt install  libffi-dev libbz2-dev libncursesw5-dev libgdbm-dev liblzma-dev libsqlite3-dev tk-dev uuid-dev libreadline-dev

# 从 github 克隆代码
$ git clone https://github.com/python/cpython.git
# 切换到 v3.10.4
$ git checkout v3.10.4
# 查看是否切换成功,如下所示就表示成功切换到了 v3.10.4 的代码
$ git branch
* (HEAD detached at v3.10.4)
  main
\end{lstlisting}

第二步就是进行代码的编译

\begin{lstlisting}[language=bash, numbers=left, numbersep=1em, numberstyle=\footnotesize , breaklines=true]
# 进入 cpython 代码目录
cd cpython
# 根据当前系统环境,生成带 debug 信息的 Makefile 
$ ./configure --with-pydebug
# 编译代码
$ make -j8
\end{lstlisting}

上面编译过程中使用的默认优化参数是 -g -Og,如果希望修改成零优化,可以修改生成的 Makefile 文件,将 OPT= -g -O0 -Wall
修改为 OPT= -g -O0 -Wall。这次修改会在下次重新执行 ./configure 之前保持有效,因为每次执行 ./configure 都会重新生成
Makefile 文件。

在编译完成之后,可以在当前目录看到一个叫做 python.exe 的文件(在 Linux 系统下,编译出来的可执行文件名是 python),
这个文件就是我们正常使用的 Python 程序了。

接下来,你可以使用这个程序进行一些 Python 代码的测试。

\begin{lstlisting}[language=bash, numbers=left, numbersep=1em, numberstyle=\footnotesize , breaklines=true]
$ ./python.exe
Python 3.10.4 (tags/v3.10.4:9d38120e33, May 15 2022, 02:03:06) [Clang 13.0.0 (clang-1300.0.29.30)] on darwin
Type "help", "copyright", "credits" or "license" for more information.
>>> print("test print")
test print
>>> print(2**3)
8
\end{lstlisting}


\begin{definition}[编译过程中的警告] \label{def:int}
在编译的过程中,如果提示 ssh、 \_lzma 等少数几个模块没有 build 成功的话也没有关系,不会影响后面的讲解。只要 python.exe 文件
成功编译出来即可。也可以通过在终端最后几行的输出中查找是否有 "Python build finished successfully!" 字样来判断是否编译成功。
\end{definition}


\section{文档编译}

在学习 cpython 源代码的过程中,如果能阅读相关模块的设计文档可以起到事半功倍的作用,所幸 cpython 考虑到了这一点,为我们准备好了丰富的
文档来辅助学习。cpython 的文档都放置在 cpython/Doc 路径下,以 rst 文件的纯文本格式保存。如果不想直接看 rst 文件,可以将文档编译为其他
的格式。下面就是 cpython 文档支持的转换格式。

\begin{lstlisting}[language=bash, numbers=left, numbersep=1em, numberstyle=\footnotesize , breaklines=true]
$ make
Please use `make <target>' where <target> is one of
  clean      to remove build files
  venv       to create a venv with necessary tools
  html       to make standalone HTML files
  htmlview   to open the index page built by the html target in your browser
  htmlhelp   to make HTML files and a HTML help project
  latex      to make LaTeX files, you can set PAPER=a4 or PAPER=letter
  text       to make plain text files
  texinfo    to make Texinfo file
  epub       to make EPUB files
  changes    to make an overview over all changed/added/deprecated items
  linkcheck  to check all external links for integrity
  coverage   to check documentation coverage for library and C API
  doctest    to run doctests in the documentation
  pydoc-topics  to regenerate the pydoc topics file
  dist       to create a "dist" directory with archived docs for download
  suspicious to check for suspicious markup in output text
  check      to run a check for frequent markup errors
  serve      to serve the documentation on the localhost (8000)
\end{lstlisting}

对通常的开发者来说,编译为 HTML 就足够了,可以在阅读源代码的同时使用浏览器查看对应的文档。编译为 HTML 的方式也很简单,只需要
在 cpython/Doc 目录下执行 make htmlview 即可,编译出的 HTML 格式文档保存在 cpython/Doc/build/html 目录下,使用浏览器打开这个
目录下的 index.html 文件即可看到文档内容。

\begin{definition}[rst 文件] \label{def:int}
rst 是 reStructuredText 的缩写,reStructuredText 是一种易于阅读,所见即所得的纯文本标记语法和解析器系统。reStructuredText 的主要目标
是定义和实现用于 Python 文档字符串和其他文档域的标记语法,该语法可读且简单,但足够强大,可以轻松使用。通俗来说,reStructuredText
 是一种和 markdowm 类似的标记语法,rst 文件是一种和 markdown 文档类似的纯文本文档。
\end{definition}

\section{dis 工具}

本节介绍下 dis 工具的使用。

\section{静态值工具}

本节介绍开发一个分析静态数据的工具。

\section{参考资料}

\begin{itemize}
\item \href{https://peps.python.org/pep-0006/}{PEP 6 – Bug Fix Releases}
\item \href{https://docutils.sourceforge.io/rst.html}{reStructuredText-Markup Syntax and Parser Component of Docutils}
\end{itemize}
