%%%%%%%%%%%%%%%%%%%%%%%%%%%%%%%%%%%%%%%%%%%%%%%%%%%%
%%% Python 字节码
%%%%%%%%%%%%%%%%%%%%%%%%%%%%%%%%%%%%%%%%%%%%%%%%%%%%

\chapter{Python 字节码}

\section{line number table}

Python 需要一个对照表结构来表明字节码和源代码的对照关系。最简单来说,要实现这样子一张对照表,
需要对每行 Python 源代码需要三个字段。字节码开始位置,字节码结束位置,该段字节码对应的源代码
行,需要源代码行的原因是源代码并不是每一行都有内容,而是可能会存在空行。

考虑如下最简单的对照表

\begin{lstlisting}[language=C, numbers=left, numbersep=1em, numberstyle=\footnotesize , breaklines=true]
     Start    End     Line
      0       6       1
      6       50      2
      50      350     7
      350     360     No line number
      360     376     8
      376     380     208
\end{lstlisting}

如果 Start, End, Line 每一个字段都用 4 字节来表示的话,那么每一行就需要 12 字节,这个内存消耗显然
太大了,所以需要压缩存储对照表使用的内存空间。可以发现,上表中,每一行的 End 等于下一行的 Start,
每一行的 Line 也可以改为相对上一行 Line 的偏移(第一行的 Line 一定是相对于 0 来计算的),那么每一行
就可以改为两列 (End - Start, Line 相对上一行的偏移),所以可以得出下面这个偏移表。

\begin{lstlisting}[language=C, numbers=left, numbersep=1em, numberstyle=\footnotesize , breaklines=true]
   End-Start  Line-delta
      6         +1
      44        +1
      300       +5
      10        No line number
      16        +1
      4         +200
\end{lstlisting}

这个更改减少了一列,节约了 $\frac{1}{3}$ 的空间。但是还可以继续优化,在实际的代码中,我们发现,End - Start
 和 Line 相对上一行的偏移大部分情况下都是一个小整数,所以,End - Start 和 Line-delta 都应该可以优化为用一个字节
 表示。End-Start 可以优化为一个 [0, 254] 的整数,Line-delta 可以优化为 [-128, 127] 的整数,Line-delta 的 -128 用来表示
 这一列没有行号,0 表示这一列没有字节码(实质上是用来分解上一列的 End-Start 的,因为每一列的 End-Start 允许的
 最大值为 254,超过了这个值,就需要分解为多行)。在这个限制条件下,可以得出第三种对照表,这个对照表就是内存
 中最后实际存储的对照表。在这个条件下,对照表每一列只需要两字节的存储空间,在最优的情况下,相比于原版每一列
 12 字节,节约了 $\frac{5}{6}$ 的空间。
 
\begin{lstlisting}[language=C, numbers=left, numbersep=1em, numberstyle=\footnotesize , breaklines=true]
    Start delta   Line delta
    6               +1
    44              +1
    254             +5
    46              0     // 上一列的 START delta 为 300,超过了 254 的上限,所以需要分解为 254 + 46 两列
    10              -128 (No line number, treated as a delta of zero)
    16              +1
    0               +127 (line 135, but the range is empty as no bytecodes are at line 135)
    4               +73 // 上一列的 Line delta 为 200,超过了 127 的上限,所以需要分解为 127 + 73 两列
\end{lstlisting}














