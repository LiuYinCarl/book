\chapter{系统属性查看}

本章节主要介绍一些查看系统属性、文件熟悉的Linux程序。

\section{file}

file 程序会测试每个参数来尝试对文件进行分类。按如下顺序执行三组测试:文件系统测试、魔术测试、
语言测试。按这个顺序,哪个测试成功了,就打印对应的文件类型。

打印的文件类型通常为下面几类:
\begin{itemize}
    \item text 文本类型,只包含可打印字符和一些控制字符。
    \item executable 可执行文件类型。
    \item data 数据类型,表示除上面两类外的其他类型,文件内容通常是二进制或不可打印的格式。
\end{itemize}

file 命令至少从 Research Version 4 版本就被添加到了 UNIX 系统中,System V 系统版本引入了
魔法类型的外部列表,稍微减慢了 file 程序的速度但是使得程序更加灵活。

版本:file-5.38

\subsection{用法和参数}

\begin{lstlisting}[language=bash, numbersep=1em, numberstyle=\footnotesize , breaklines=true]
file [-bcdEhiklLNnprsSvzZ0] [--apple] [--exclude-quiet]
      [--extension] [--mime-encoding] [--mime-type] [-e testname]
      [-F separator] [-f namefile] [-m magicfiles] [-P name=value]
      file ...

     -b, --brief
	     不要将文件名添加到输出行(简要模式)

     -i, --mime
             输出文件的 mime type 字符串。

     --mime-type, --mime-encoding
             和 -i 类似,但是只打印指定类型的元素
             
     -p, --preserve-date
             在支持 utime(3) 或者 utimes(2) 的系统上,让 file 命令在执行的
             时候不修改文件的访问时间

     -r, --raw
             不将不可大打印的字符转换成八进制,一般而言,file 指令会将不可打印
             的字符转换成八进制表示
             
     -0, --print0
             在文件名之后输出一个 '\0' 字符,便于使用 cut 指令进行分割,这个
             参数不会影响分隔符的正常输出

             如果这个参数被重复多次,file 会在输出的每个字段之后都加上一个 NUL('\0')

\end{lstlisting}

\subsection{使用示例}
使用到的测试文件如下。

file.txt
\begin{lstlisting}[language=bash, numbersep=1em, numberstyle=\footnotesize , breaklines=true]
file cmd test.
\end{lstlisting}

各参数使用例子。
\begin{lstlisting}[language=bash, numbersep=1em, numberstyle=\footnotesize , breaklines=true]
$ file file.txt
file.txt: ASCII text

$ file -b file.txt
ASCII text

$ file -i file.txt
file.txt: text/plain; charset=us-ascii

$ file --mime-encoding file.txt
file.txt: us-ascii

$ file file.txt | cut -d '' -f 1
file.txt: ASCII text

$ file file.txt -0 | cut -d '' -f 1
file.txt
\end{lstlisting}


\subsection{文档}
\begin{itemize}
\item \href{https://man7.org/linux/man-pages/man1/file.1.html}{在线手册}
\end{itemize}
