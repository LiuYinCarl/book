\chapter{文本操作程序}

本章节介绍常用的 Linux 文本操作程序。

\section{cut}

cut 是一个文本处理程序, 用来取出每个文件中的每一行的某个部分。

版本:cut (GNU coreutils) 8.30

\subsection{用法和参数}

\begin{lstlisting}[language=bash, numbersep=1em, numberstyle=\footnotesize , breaklines=true]
cut OPTION [FILE]
          OPTION 参数是必选的,FILE 参数是可选的

   -b, --bytes=LIST
          选择特定的字节序列

   -c, --characters=LIST
          选择特定的字符序列

   -d, --delimiter=DELIM
          使用 DELIM 而不是 TAB 作为字段分隔符,默认使用 TAB 作为字段分隔符

   -f, --fields=LIST
          选择特定的字段序列; 除非指定了 -s 参数,否则会打印所有
          不包含分隔符的行

   -n     (ignored)

   --complement
          补充所选的字节、字符或字段集

   -s, --only-delimited
          不打印不包含分隔符的行

   --output-delimiter=STRING
          使用 STRING 作为输出的字段分隔符,默认使用和输入相同的字段分隔符

   -z, --zero-terminated
          使用 NUL 而不是换行符作为行分隔符,不能单独使用
\end{lstlisting}

\subsection{使用示例}

使用的测试文件 test.txt 内容如下。
\begin{lstlisting}[language=bash, numbersep=1em, numberstyle=\footnotesize , breaklines=true]
abc cde
bdf fgh
abc hhh
\end{lstlisting}


\begin{lstlisting}[language=bash, numbersep=1em, numberstyle=\footnotesize , breaklines=true]
# 输出每行的第 1 个字节
$ cut -b 1 test.txt
a
b
a

# 输出每行的第1到3字节
$ cut -b 1-3 test.txt
abc
bdf
abc

# 输出每行的第5个字节及之后的内容
$ cut -b 5- test.txt
cde
fgh
hhh

# 输出每行的第1到3个字符, -b与-c参数的用法类似
$ cut -c 1-3 test.txt
abc
bdf
abc

# 以空格为分隔符将每一行切割为多列,输出第二列
$ cut -d " " -f 2 test.txt
cde
fgh
hhh

# 任务: 假设服务器所有日志存在一个文件
# 假设
$ find . -name "*.log" | xargs file | grep "text" | cut -d: -f1 | xargs cat | grep "on_bpass_season_end"

\end{lstlisting}

\subsection{练习}
假设服务器所有日志存储在一个目录/server\_log/下,文件名格式为module\_name.log,日志文件大小达到10M之后
会将旧日志重命名为module\_name.log.1, module\_name.log.2等,当旧日志数量达到10个之后,会将旧日志进行
压缩,压缩文件命名为module\_name.log.tar.gz,现在要求查找http模块中所有包含"UserName"的日志,压缩日志除外。

\begin{lstlisting}[language=bash, numbersep=1em, numberstyle=\footnotesize , breaklines=true]
find . -name "http.log*" | xargs file | grep "text" | cut -d: -f1 | xargs cat | grep "UserName"
\end{lstlisting}

\subsection{文档}
\begin{itemize}
\item \href{https://man7.org/linux/man-pages/man1/cut.1.html}{在线手册}
\end{itemize}

\newpage
\section{rev}

rev 程序的功能是按字符反转文本行。rev 将指定的文件复制到标准输出并且颠倒每一行中字符的顺序。如果没有
指定文件,则读取标准输入。rev 使用内为宽字符行分配的内存缓冲区,如果输入文件非常大并且没有换行符的话,
rev 程序可能运行失败。

\subsection{用法和参数}

\begin{lstlisting}[language=bash, numbersep=1em, numberstyle=\footnotesize , breaklines=true]
uniq [OPTION]... [INPUT [OUTPUT]]
rev [option] [file...]
\end{lstlisting}

\subsection{使用示例}

用来测试的文件 test.txt 内容
\begin{lstlisting}[language=bash, numbersep=1em, numberstyle=\footnotesize , breaklines=true]
abc def
这是一行测试 rev 的文本
\end{lstlisting}

\begin{lstlisting}[language=bash, numbersep=1em, numberstyle=\footnotesize , breaklines=true]
$ rev test.txt
fed cba
本文的 ver 试测行一是这
\end{lstlisting}


\subsection{文档}
\begin{itemize}
\item \href{https://www.man7.org/linux/man-pages/man1/rev.1.html}{在线手册}
\end{itemize}

\newpage
\section{uniq}

uniq 从输入文件或者标准输入中过滤\textbf{相邻}的匹配行,写入输入文件或者标准输入,
如果没有添加任何参数的话,匹配的行都会合并到首次出现的行中。uniq 在日常中主要用来做文本重复行的过滤、统计、去重三个操作。


\subsection{用法和参数}

\begin{lstlisting}[language=bash, numbersep=1em, numberstyle=\footnotesize , breaklines=true]
uniq [OPTION]... [INPUT [OUTPUT]]

   -c, --count
          统计每一行出现的次数,出现次数并作为前缀添加到首次出现的行前

   -d, --repeated
          输出所有重复的行,每组重复的行只输出一次

   -D     输出所有重复的行

   --all-repeated[=METHOD]
          和 —D 类似,但是允许使用空行来分隔不同的重复组
          METHOD={none(default),prepend,separate}

   -f, --skip-fields=N
          不比较每行的前 N 个字段

   --group[=METHOD]
          展示所有项,并且使用空行进行分割
          METHOD={separate(default),prepend,append,both}

   -i, --ignore-case
          在比较的时候忽略大小写

   -s, --skip-chars=N
          跳过比较每行的前 N 个字符

   -u, --unique
          只打印出现过一次的行

   -z, --zero-terminated
          使用 NUL 而不是换行符作为行分割符

   -w, --check-chars=N
          只比较每行的最多前 N 个字符
\end{lstlisting}

\subsection{使用示例}
下面测试中使用的文件。

uniq.txt
\begin{lstlisting}[language=bash, numbersep=1em, numberstyle=\footnotesize , breaklines=true]
aaaa
bbbb
cccc
aaaa
bbbb
aaaa
\end{lstlisting}

uniq2.txt
\begin{lstlisting}[language=bash, numbersep=1em, numberstyle=\footnotesize , breaklines=true]
1 aaaa
2 bbbb
3 cccc
4 aaaa
5 bbbb
6 aaaa
\end{lstlisting}

uniq3.txt
\begin{lstlisting}[language=bash, numbersep=1em, numberstyle=\footnotesize , breaklines=true]
aaa12
aaa23
bbb34
\end{lstlisting}

uniq 各参数使用实例。
\begin{lstlisting}[language=bash, numbersep=1em, numberstyle=\footnotesize , breaklines=true]
$ cat uniq.txt | sort | uniq
aaaa
bbbb
cccc

$ cat uniq.txt | sort | uniq -c
      3 aaaa
      2 bbbb
      1 cccc
      
$ cat uniq.txt | sort | uniq -d
aaaa
bbbb

$ cat uniq.txt | sort | uniq -D
aaaa
aaaa
aaaa
bbbb
bbbb

$ cat uniq.txt | sort | uniq --all-repeated=separate
aaaa
aaaa
aaaa

bbbb
bbbb

$ cat uniq2.txt | rev | sort | rev | uniq --skip-fields=1 -D
1 aaaa
4 aaaa
6 aaaa
2 bbbb
5 bbbb

$ cat uniq.txt | sort | uniq --group=append
aaaa
aaaa
aaaa

bbbb
bbbb

cccc

$ cat uniq2.txt | rev | sort | rev | uniq -s 2 -D
1 aaaa
4 aaaa
6 aaaa
2 bbbb
5 bbbb

$ cat uniq.txt | sort | uniq -u
cccc

$ cat uniq3.txt | uniq -w 3 -c
      2 aaa12
      1 bbb34
\end{lstlisting}

\subsection{文档}
