\documentclass[lang=cn,10pt]{elegantbook}

%%%%%%%%%%%%%%%%%%%%%%%%%%%%%%%%%%%%%%%%%%%%%%%%%%%%%%%%%%%%%%%%%%%%%%%%%%%%%%
%% 包设置
%%%%%%%%%%%%%%%%%%%%%%%%%%%%%%%%%%%%%%%%%%%%%%%%%%%%%%%%%%%%%%%%%%%%%%%%%%%%%%
\usepackage[ruled,noline]{algorithm2e} % 算法包
%\setlength{\lineskip}{2em} %设置行间距
\setlength{\parskip}{0.7em} %设置段落间距
% 设置代码框内的行间距
\AtBeginEnvironment{lstlisting}{\linespread{1.0}}

\lstset{
    basicstyle = \ttfamily,           % 基本样式 + 小号字体
    breaklines = true,                  % 代码过长则换行
    frame = shadowbox,                  % 用(带影子效果)方框框住代码块
    showspaces = false,                 % 不显示空格
    columns = fixed,                    % 字间距固定
    numbers=left,                       % 左侧展示行号
    numberstyle=\it,                    % 斜体行号
    escapeinside={@(}{)@},  % 设置 escape 字符,放置在 @( )@ 中间的内容会被当作 LaTex 进行解释,例如 @($\alpha$)@
}

% 本文档命令
\usepackage{array}
\newcommand{\ccr}[1]{\makecell{{\color{#1}\rule{1cm}{1cm}}}}

%%%%%%%%%%%%%%%%%%%%%%%%%%%%%%%%%%%%%%%%%%%%%%%%%%%%%%%%%%%%%%%%%%%%%%%%%%%%%%
%% 封面设置
%%%%%%%%%%%%%%%%%%%%%%%%%%%%%%%%%%%%%%%%%%%%%%%%%%%%%%%%%%%%%%%%%%%%%%%%%%%%%%
\title{编译原理笔记}
\author{kenshinl}
\date{Last Update: \today}
\version{0.1}
\bioinfo{封面图片作者}{Elina Bernpaintner}
\extrainfo{人生代代无穷已,江月年年望相似。}
\setcounter{tocdepth}{3}
\cover{cover.jpg}
% 修改标题页的橙色带
\definecolor{customcolor}{RGB}{255, 255, 255}
\colorlet{coverlinecolor}{customcolor}


\begin{document}
\maketitle
\frontmatter
\tableofcontents
\mainmatter


\chapter{环境准备}

本书对 Python 3.10.4 版本的代码实现进行探究。使用的 Python 实现为 CPython。

\section{Python 编译}

为了了解 Python 的细节,探究和验证代码的实现是否符合自己的预期,我们需要亲自动手修改源码,添加测试代码。
为了完成这个目的,需要学会自己动手编译 Python 源码。所以在本节,将介绍如何编译 Python 代码。

\subsection{在 Linux/MacOS 平台上编译 Python}

编译步骤如下:

第一步是下载 Python 代码并切换到 Python 3.10.4 版本。

\begin{lstlisting}[language=bash, numbers=left, numbersep=1em, numberstyle=\footnotesize , breaklines=true]
# 从 github 克隆代码
$ git clone https://github.com/python/cpython.git
# 切换到 v3.10.4
$ git checkout v3.10.4
# 查看是否切换成功,如下所示就表示成功切换到了 v3.10.4 的代码
$ git branch
* (HEAD detached at v3.10.4)
  main
\end{lstlisting}

第二步就是进行代码的编译

\begin{lstlisting}[language=bash, numbers=left, numbersep=1em, numberstyle=\footnotesize , breaklines=true]
# 进入 cpython 代码目录
cd cpython
# 根据当前系统环境,生成带 debug 信息的 Makefile 
$ ./configure --with-pydebug
# 编译代码
$ make -j8
\end{lstlisting}

在编译完成之后,可以在当前目录看到一个叫做 python.exe 的文件(在 Linux 系统下,编译出来的可执行文件名是 python),
这个文件就是我们正常使用的 Python 程序了。

接下来,你可以使用这个程序进行一些 Python 代码的测试。

\begin{lstlisting}[language=bash, numbers=left, numbersep=1em, numberstyle=\footnotesize , breaklines=true]
$ ./python.exe
Python 3.10.4 (tags/v3.10.4:9d38120e33, May 15 2022, 02:03:06) [Clang 13.0.0 (clang-1300.0.29.30)] on darwin
Type "help", "copyright", "credits" or "license" for more information.
>>> print("test print")
test print
>>> print(2**3)
8
\end{lstlisting}


\begin{definition}[编译过程中的警告] \label{def:int}
在编译的过程中,如果提示 ssh、 \_lzma 等少数几个模块没有 build 成功的话也没有关系,不会影响后面的讲解。只要 python.exe 文件
成功编译出来即可。也可以通过在终端最后几行的输出中查找是否有 "Python build finished successfully!" 字样来判断是否编译成功。
\end{definition}


\section{文档编译}

在学习 cpython 源代码的过程中,如果能阅读相关模块的设计文档可以起到事半功倍的作用,所幸 cpython 考虑到了这一点,为我们准备好了丰富的
文档来辅助学习。cpython 的文档都放置在 cpython/Doc 路径下,以 rst 文件的纯文本格式保存。如果不想直接看 rst 文件,可以将文档编译为其他
的格式。下面就是 cpython 文档支持的转换格式。

\begin{lstlisting}[language=bash, numbers=left, numbersep=1em, numberstyle=\footnotesize , breaklines=true]
$ make
Please use `make <target>' where <target> is one of
  clean      to remove build files
  venv       to create a venv with necessary tools
  html       to make standalone HTML files
  htmlview   to open the index page built by the html target in your browser
  htmlhelp   to make HTML files and a HTML help project
  latex      to make LaTeX files, you can set PAPER=a4 or PAPER=letter
  text       to make plain text files
  texinfo    to make Texinfo file
  epub       to make EPUB files
  changes    to make an overview over all changed/added/deprecated items
  linkcheck  to check all external links for integrity
  coverage   to check documentation coverage for library and C API
  doctest    to run doctests in the documentation
  pydoc-topics  to regenerate the pydoc topics file
  dist       to create a "dist" directory with archived docs for download
  suspicious to check for suspicious markup in output text
  check      to run a check for frequent markup errors
  serve      to serve the documentation on the localhost (8000)
\end{lstlisting}

对通常的开发者来说,编译为 HTML 就足够了,可以在阅读源代码的同时使用浏览器查看对应的文档。编译为 HTML 的方式也很简单,只需要
在 cpython/Doc 目录下执行 make htmlview 即可,编译出的 HTML 格式文档保存在 cpython/Doc/build/html 目录下,使用浏览器打开这个
目录下的 index.html 文件即可看到文档内容。

\begin{definition}[rst 文件] \label{def:int}
rst 是 reStructuredText 的缩写,reStructuredText 是一种易于阅读,所见即所得的纯文本标记语法和解析器系统。reStructuredText 的主要目标
是定义和实现用于 Python 文档字符串和其他文档域的标记语法,该语法可读且简单,但足够强大,可以轻松使用。通俗来说,reStructuredText
 是一种和 markdowm 类似的标记语法,rst 文件是一种和 markdown 文档类似的纯文本文档。
\end{definition}

\section{dis 工具}

本节介绍下 dis 工具的使用。

\section{静态值工具}

本节介绍开发一个分析静态数据的工具。

\section{参考资料}

\begin{itemize}
\item \href{https://peps.python.org/pep-0006/}{PEP 6 – Bug Fix Releases}
\item \href{https://docutils.sourceforge.io/rst.html}{reStructuredText-Markup Syntax and Parser Component of Docutils}
\end{itemize}
 % 《编译原理 第二版》笔记
\chapter{一些概念和问题}

\section*{First集和Follow集的作用}


\subsection*{算法实现}

\begin{algorithm}[H]
    \renewcommand{\thealgocf}{}
    \caption{\texttt{求nullable、FIRST 集和 FOLLOW 集}}
%    \KwIn{None}
%    \KwOut{None}
将所有的 FIRST 和 FOLLOW 集合初始化为空,将所有的 nullable 初始化为 false \\
\For {每一个终结符 $Z$} {
	$FIRST[Z] \leftarrow \{Z\}$
}

\Repeat {$FIRST$ 、$FOLLOW$ 和 $nullable$ 在此轮迭代中没有改变} {
\For {每个产生式 $X \rightarrow Y_{1}Y_{2} \cdots Y_{k}$} {
	\For {每个 $i$ 从 $1$ 到 $k$,每个 $j$ 从$i+1$到$k$} {
		\If {所有 $Y_{i}$都是可为空的}{
			$nullable[X] \leftarrow true$ \\
		}
		\If {$Y_{1} \cdots Y_{i-1}$都是可为空的}{
			$FIRST[X] \leftarrow FIRST[X] \cup FIRST[Y_{i}]$ \\
		}
		\If {$Y_{i+1} \cdots Y_{k}$ 都是可为空的}{
			$FOLLOW[Y_{i}] \leftarrow FOLLOW[Y_{i}] \cup FOLLW[X]$ \\
		}
		\If {$Y_{i+1} \cdots Y_{j-1}$都是可为空的}{
			$FOLLOW[Y_{i}] \leftarrow FOLLOW[Y_{i}] \cup FIRST[Y_{j}]$ \\
		}
	}
}
}

\end{algorithm}



\section*{词法分析和语法分析的关联体现在什么地方}


\section*{左递归带来的问题是什么}

要搞清楚这个问题,首先需要知道什么是左递归。左递归的定义如下:

对于上下文无关文法的一个规则来说,如果其右侧第一个符号与左侧符号相同或者能够推导出左侧符号,
那么称该规则为左递归的。前一种情况称为直接左递归,后一种情况成为间接左递归。



以一个经典的包含左递归的表达式语法为例。

\begin{lstlisting}[caption={经典表达式文法}]
Goal   -> Expr
Expr   -> Expr + Term
Expr   |  Expr - Term
       |  Term
Term   -> Term * Factor
Term   |  Term / Factor
       |  Factor
Factor -> ( Expr )
       |  num
       | name
\end{lstlisting}

左递归是自顶向下语法分析中需要专门处理的一个问题。自定向下语法语法分析是指从语法分析树的根开始,
系统地向下拓展树,直至树的叶结点与词法分析器返回地已归类单词相匹配。在过程的每一点上,都需要考虑
一个部分完成的语法分析树。过程在树的下边缘选择一个非终结符,选定某个适用于该非终结符的产生式,用
与产生式右侧相对应的子树来拓展该结点。终结符是无法拓展的。这个过程会一直持续下去,直到语法分析树
的下边缘只包含终结符,且输入流已经耗尽。

从上面的描述中可以看出,自顶向下语法分析的关键就是选择一个合适的非终结符进行拓展。以句子
a+b*c 为例,了解左递归对自顶向下语法分析带来的问题。

\begin{lstlisting}[caption={自顶向下语法分析 a+b*c}]
规则0  Expr
规则1  Expr + Term
规则1  Expr + Term + Term
规则1  Expr + Term + Term + Term
...
\end{lstlisting}

在第二行开始,语法分析器知道非终结符 Expr 和输入单词 a,它选择规则1进行匹配,在第三行,语法分析器
还是面临非终结符 Expr 和输入单词 a,他继续选择规则1进行匹配,导致了无限循环。这也就是左递归给自顶向下
语法分析带来的问题。在这个例子中,语法分析器向前多看一个单词在这里可以解决问题,但是如果语法足够复杂,
左递归仍然会导致这个问题。所以,需要有一个算法来消除左递归。对于直接左递归,可以采用如下的方式。

\begin{lstlisting}[caption={消除直接左递归}]
Fee -> Fee @($\alpha$)@
    |  @($\beta$)@
    
转换为如下产生式

Fee  -> @($\beta$)@ Fee'
Fee' -> @($\alpha$)@ Fee'
     |  @($\epsilon$)@
\end{lstlisting}

对于间接左递归的消除算法就不在这里描述了。

\section*{什么是递归下降分析器}
 % 一些概念和问题

\end{document}