\chapter{一些概念和问题}

\section*{First集和Follow集的作用}


\subsection*{算法实现}

\begin{algorithm}[H]
    \renewcommand{\thealgocf}{}
    \caption{\texttt{求nullable、FIRST 集和 FOLLOW 集}}
%    \KwIn{None}
%    \KwOut{None}
将所有的 FIRST 和 FOLLOW 集合初始化为空,将所有的 nullable 初始化为 false \\
\For {每一个终结符 $Z$} {
	$FIRST[Z] \leftarrow \{Z\}$
}

\Repeat {$FIRST$ 、$FOLLOW$ 和 $nullable$ 在此轮迭代中没有改变} {
\For {每个产生式 $X \rightarrow Y_{1}Y_{2} \cdots Y_{k}$} {
	\For {每个 $i$ 从 $1$ 到 $k$,每个 $j$ 从$i+1$到$k$} {
		\If {所有 $Y_{i}$都是可为空的}{
			$nullable[X] \leftarrow true$ \\
		}
		\If {$Y_{1} \cdots Y_{i-1}$都是可为空的}{
			$FIRST[X] \leftarrow FIRST[X] \cup FIRST[Y_{i}]$ \\
		}
		\If {$Y_{i+1} \cdots Y_{k}$ 都是可为空的}{
			$FOLLOW[Y_{i}] \leftarrow FOLLOW[Y_{i}] \cup FOLLW[X]$ \\
		}
		\If {$Y_{i+1} \cdots Y_{j-1}$都是可为空的}{
			$FOLLOW[Y_{i}] \leftarrow FOLLOW[Y_{i}] \cup FIRST[Y_{j}]$ \\
		}
	}
}
}

\end{algorithm}



\section*{词法分析和语法分析的关联体现在什么地方}


\section*{左递归带来的问题是什么}

要搞清楚这个问题,首先需要知道什么是左递归。左递归的定义如下:

对于上下文无关文法的一个规则来说,如果其右侧第一个符号与左侧符号相同或者能够推导出左侧符号,
那么称该规则为左递归的。前一种情况称为直接左递归,后一种情况成为间接左递归。



以一个经典的包含左递归的表达式语法为例。

\begin{lstlisting}[caption={经典表达式文法}]
Goal   -> Expr
Expr   -> Expr + Term
Expr   |  Expr - Term
       |  Term
Term   -> Term * Factor
Term   |  Term / Factor
       |  Factor
Factor -> ( Expr )
       |  num
       | name
\end{lstlisting}

左递归是自顶向下语法分析中需要专门处理的一个问题。自定向下语法语法分析是指从语法分析树的根开始,
系统地向下拓展树,直至树的叶结点与词法分析器返回地已归类单词相匹配。在过程的每一点上,都需要考虑
一个部分完成的语法分析树。过程在树的下边缘选择一个非终结符,选定某个适用于该非终结符的产生式,用
与产生式右侧相对应的子树来拓展该结点。终结符是无法拓展的。这个过程会一直持续下去,直到语法分析树
的下边缘只包含终结符,且输入流已经耗尽。

从上面的描述中可以看出,自顶向下语法分析的关键就是选择一个合适的非终结符进行拓展。以句子
a+b*c 为例,了解左递归对自顶向下语法分析带来的问题。

\begin{lstlisting}[caption={自顶向下语法分析 a+b*c}]
规则0  Expr
规则1  Expr + Term
规则1  Expr + Term + Term
规则1  Expr + Term + Term + Term
...
\end{lstlisting}

在第二行开始,语法分析器知道非终结符 Expr 和输入单词 a,它选择规则1进行匹配,在第三行,语法分析器
还是面临非终结符 Expr 和输入单词 a,他继续选择规则1进行匹配,导致了无限循环。这也就是左递归给自顶向下
语法分析带来的问题。在这个例子中,语法分析器向前多看一个单词在这里可以解决问题,但是如果语法足够复杂,
左递归仍然会导致这个问题。所以,需要有一个算法来消除左递归。对于直接左递归,可以采用如下的方式。

\begin{lstlisting}[caption={消除直接左递归}]
Fee -> Fee @($\alpha$)@
    |  @($\beta$)@
    
转换为如下产生式

Fee  -> @($\beta$)@ Fee'
Fee' -> @($\alpha$)@ Fee'
     |  @($\epsilon$)@
\end{lstlisting}

对于间接左递归的消除算法就不在这里描述了。

\section*{什么是递归下降分析器}
